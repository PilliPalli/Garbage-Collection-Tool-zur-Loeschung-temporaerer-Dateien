
\subsubsection*{Muss-Kriterien (essenziell für die Projektabnahme)}

\begin{itemize}
  \item \textbf{Effektive Datenbereinigung:} Das Tool muss temporäre und unnötige Dateien zuverlässig identifizieren und entfernen können.

  \item \textbf{Übereinstimmung mit dem Corporate-Design:} Die Benutzeroberfläche der Software muss den visuellen Richtlinien des Unternehmens entsprechen.

  \item \textbf{Kompatibilität:} Die Anwendung muss auf den im Unternehmen eingesetzten Systemen lauffähig sein und stabil arbeiten.

  \item \textbf{Benutzerdefinierte Einstellungen:} Benutzer müssen in der Lage sein, den Bereinigungsprozess über eine Konfigurationsdatei oder GUI-Optionen individuell anzupassen.
\end{itemize}

\vspace{1em}
\subsubsection*{Kann-Kriterien (wünschenswert, aber nicht zwingend für die Abnahme)}

\begin{itemize}
  \item \textbf{Erweiterte Berichtsfunktionen:} Ausgabe detaillierter Informationen zu den durchgeführten Löschvorgängen und der Menge des freigegebenen Speicherplatzes.

  \item \textbf{Echtzeit-Monitoring und Warnungen:} Anzeige des aktuellen Bereinigungsstatus sowie Benachrichtigungen bei Problemen (wie zum Beispiel, wenn keine Dateien gefunden wurden).

  \item \textbf{Log-Export:} Möglichkeit zum Export der Bereinigungsprotokolle in ein CSV-Format.

  \item \textbf{Duplikaterkennung und -entfernung:} Identifikation und optionales Entfernen von mehrfach vorhandenen Dateien durch Hashvergleich.
\end{itemize}
