Für die Speicherung von Benutzerdaten, Rollen, Protokollen sowie Konfigurationen wird eine relationale Datenbank auf Basis von \textbf{Microsoft SQL Server} eingesetzt. Die Anbindung erfolgt über \textbf{Entity Framework Core}, wodurch eine objektorientierte Kommunikation mit der Datenbank möglich ist. Die Datenbankmodelle – beispielsweise \texttt{User}, \texttt{CleanupLog} oder \texttt{Role} – sind als C\#-Klassen im Projekt umgesetzt und werden über einen zentralen \texttt{DbContext} verwaltet.

\subsubsection*{Entwicklungsumgebung mit Docker}

Während der Entwicklung kommt \textbf{Docker} zum Einsatz, um die Datenbank in einem isolierten Container bereitzustellen. Dies bietet mehrere Vorteile:

\begin{itemize}
    \item \textbf{Plattformunabhängigkeit:} Die Umgebung lässt sich auf jedem System mit Docker identisch starten.
    \item \textbf{Reproduzierbarkeit:} Fehler durch unterschiedliche Systemkonfigurationen werden minimiert.
    \item \textbf{Schnelles Setup:} Die gesamte Datenbank ist mit einem einzigen Befehl einsatzbereit.
    \item \textbf{Isolierte Testumgebung:} Änderungen an der Datenbankstruktur können risikolos getestet werden.
\end{itemize}

\begin{figure}[H]
    \centering
    \begin{shellcode}
sudo docker run -e "ACCEPT_EULA=Y" -e "MSSQL_SA_PASSWORD=VeryStr0ngP@ssw0rd" --name sql -p 1433:1433 -v sql_server:/var/opt/mssql -d --restart=always --hostname sql --platform linux/amd64 -d mcr.microsoft.com/mssql/server:2022-latest
\end{shellcode}
    \caption{Beispiel für einen \texttt{docker run}-Befehl zur Bereitstellung eines SQL-Servers}
\end{figure}
% \begin{figure}[H]
%     \centering
%     \includegraphics[width=1\textwidth]{src/docker_code.png}
%     \caption{Beispiel für einen \texttt{docker run}-Befehl zur Bereitstellung eines SQL-Servers}
% \end{figure}

\subsubsection*{Praxisnähe durch serverseitige Architektur}

Auch im Unternehmenskontext liegt die Datenbank typischerweise nicht auf dem Client, sondern auf einem zentralen Server innerhalb des Netzwerks. Durch diese Trennung von Anwendung und Datenhaltung:

\begin{itemize}
    \item wird die Datensicherheit erhöht,
    \item ist eine zentrale Sicherung und Verwaltung möglich,
    \item können mehrere Benutzer gleichzeitig auf die Daten zugreifen.
\end{itemize}

Der Einsatz von Docker bildet diese Struktur realitätsnah ab und erleichtert die spätere Überführung in eine produktive Infrastruktur.

\subsubsection*{Connection-String-Konfiguration}
Der Connection String zur Datenbank ist in einer \texttt{config.json}-Datei gespeichert. In realen Unternehmensszenarien würde dieser sensible Wert jedoch nicht für den Endnutzer sichtbar abgelegt, sondern zentral verwaltet.

Für Demonstrationszwecke und zur Erleichterung der Konfiguration wurde in diesem Projekt bewusst auf die Lösung mittels \texttt{config.json} zurückgegriffen.
