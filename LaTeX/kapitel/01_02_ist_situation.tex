Mit zunehmendem Wachstum und der Erweiterung ihrer Dienstleistungen stößt die \textbf{DataFlow Solutions GmbH} auf ein zentrales Problem:

Die Server und Rechner des Unternehmens sammeln im Laufe der Zeit eine große Menge an temporären und überflüssigen Dateien an, was zu einer spürbaren Verlangsamung der Systeme führt. Die IT-Infrastruktur leidet unter verminderter Leistungsfähigkeit und Zuverlässigkeit. Dies äußert sich unter anderem in verlängerten Ladezeiten bei Datenanalysen und beeinträchtigt die Effizienz der Cloud-Dienste – mit potenziell negativen Folgen für die Kundenzufriedenheit und damit einhergehenden finanziellen Verlusten.

Um dieser Entwicklung entgegenzuwirken, plant DataFlow Solutions die Einführung eines \textit{Garbage-Collection-Tools}, das automatisch temporäre und unnötige Dateien erkennt und entfernt. Ziel ist es, die Systemleistung nachhaltig zu optimieren und gleichzeitig die Effizienz der Infrastruktur zu verbessern.

Durch die Implementierung dieser Lösung soll nicht nur die interne IT-Performance gesteigert, sondern auch die Kundenzufriedenheit erhöht und die operative Exzellenz des Unternehmens langfristig gesichert werden. Zudem erhofft sich DataFlow Solutions eine Stärkung seiner Position im zunehmend wettbewerbsintensiven Markt für Datenanalyse- und Cloud-Dienstleistungen.
