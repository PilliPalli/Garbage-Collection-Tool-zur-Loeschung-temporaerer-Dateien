
\addcontentsline{toc}{section}{Vorwort}

In einer zunehmend digitalisierten Welt werden analoge Prozesse durch Softwarelösungen abgebildet, automatisiert und optimiert. Die Relevanz digitaler Anwendungen hat in den vergangenen Jahren stark zugenommen. 

Auch bei \textbf{DataFlowSolutions} wird täglich mit großen Datenmengen gearbeitet, die durch verschiedenste betriebliche Prozesse entstehen. Neben relevanten und dauerhaft benötigten Informationen sammeln sich jedoch zunehmend temporäre oder überholte Daten an – sogenannte „Datenreste“ oder „Datenmüll“. Diese Dateien verlieren nach ihrer Nutzung schnell an Bedeutung, belegen jedoch weiterhin wertvollen Speicherplatz und beeinträchtigen langfristig die Systemleistung.

Aus dieser Problematik heraus entstand die Idee, ein maßgeschneidertes \sloppy Garbage-Collection-Tool zu entwickeln, das diese temporären Dateien identifiziert und automatisiert bereinigt. Ziel war es, ein benutzerfreundliches Werkzeug zu schaffen, das flexibel auf unterschiedliche Dateitypen und Verzeichnisse reagieren kann, sowohl manuell als auch zeitgesteuert.

Mit dieser Hausarbeit wird nicht nur ein konkretes Softwareprodukt dokumentiert, sondern auch ein praxisnahes Problem adressiert, das in vielen Unternehmen von großer Bedeutung ist.
