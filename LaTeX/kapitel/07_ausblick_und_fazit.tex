Mit dem entwickelten Garbage-Collection-Tool hat das Unternehmen \textit{DataFlow Solutions} eine leistungsfähige Lösung geschaffen, um seine IT-Infrastruktur nachhaltig zu entlasten. Temporäre und überflüssige Dateien werden zuverlässig erkannt und entfernt, wodurch Systemressourcen geschont, die Lebensdauer der Hardware verlängert und Betriebskosten eingespart werden können.

Dank der Architektur auf Basis des MVVM-Patterns ist das Tool flexibel erweiterbar und zukunftssicher. Perspektivisch lassen sich weitere Funktionen integrieren, wie zum Beispiel intelligentere Algorithmen zur Dateianalyse, eine automatische E-Mail-Benachrichtigung bei kritischen Datenmengen oder eine erweiterte Benutzer- und Rollenkontrolle.

Im Verlauf dieser Arbeit konnte ich nicht nur ein konkretes Softwareprodukt entwickeln, sondern auch wertvolle Erfahrungen im Bereich der Softwarearchitektur sammeln. Besonders das Zusammenspiel von Benutzeroberfläche und Backend sowie die konsequente Trennung von Logik und Darstellung haben mein Verständnis für nachhaltige Softwareentwicklung geschärft. Dieses Wissen wird mir auch bei zukünftigen Projekten und in meiner beruflichen Laufbahn als Wirtschaftsinformatiker von großem Nutzen sein.
