

\subsection*{Testfall GC01 – Automatische Datenbereinigung}
\begin{tabular}{|p{4cm}|p{10cm}|}
\hline
 \textbf{ID:} & GC01 \\
\hline
\textbf{Beschreibung:} & Überprüfung der automatischen Datenbereinigung. \\
\hline
\textbf{Vorbedingung:} & Das Tool ist installiert, konfiguriert und der Bereinigungsprozess ist aktiviert. Es existieren alte und neue Dateien. \\
\hline
\textbf{Test-Schritte:} & 
1. Es werden Testdateien in einem temporären Verzeichnis erstellt, einige davon sind älter als der angegebene Schwellenwert (z.\,B. 10 Tage). \newline
2. Das Tool wird gestartet, und der Bereinigungsprozess ausgeführt. \newline
3. Überprüfung, ob alte Dateien entfernt wurden und neue Dateien erhalten bleiben. \\
\hline
\textbf{Erwartetes \newline Resultat:} & 
- Die alten Dateien werden erfolgreich entfernt. \newline
- Die neuen Dateien bleiben bestehen. \\
\hline
\end{tabular}

\vspace{1em}

\subsection*{Testfall GC02 – Sicherheitsfunktion (Papierkorb)}
\begin{tabular}{|p{4cm}|p{10cm}|}
\hline
\textbf{ID:} & GC02 \\
\hline
\textbf{Beschreibung:} & Test der Sicherheitsfunktion, bei der Dateien in den Papierkorb verschoben werden. \\
\hline
\textbf{Vorbedingung:} & Das Tool ist installiert und konfiguriert, und die Option zum Verschieben von Dateien in den Papierkorb ist aktiviert. \\
\hline
\textbf{Test-Schritte:} & 
1. Testdateien werden in einem temporären Verzeichnis erstellt. \newline
2. Der Bereinigungsprozess wird gestartet, und überprüft, ob die Dateien in den Papierkorb verschoben werden. \newline
3. Der Papierkorb wird auf die verschobenen Dateien überprüft. \\
\hline
\textbf{Erwartetes \newline Resultat:} & 
- Die Dateien werden nicht gelöscht, sondern in den Papierkorb verschoben. \\
\hline
\end{tabular}

\vspace{1em}

\subsection*{Testfall GC03 – Benutzerdefinierte Einstellungen}
\begin{tabular}{|p{4cm}|p{10cm}|}
\hline
\textbf{ID:} & GC03 \\
\hline
\textbf{Beschreibung:} & Test der benutzerdefinierten Einstellungen. \\
\hline
\textbf{Vorbedingung:} & Das Tool ist installiert und konfiguriert, der Bereinigungsprozess ist aktiviert. \\
\hline
\textbf{Test-Schritte:} & 
1. Zugriff auf die Einstellungsoptionen des Tools. \newline
2. Anpassung der Parameter in der Konfigurationsdatei (z.B. Verzeichnis, Dateimuster, Alter der zu löschenden Dateien). \newline
3. Ausführung des Bereinigungsprozesses und Überprüfung, ob die Änderungen korrekt umgesetzt wurden. \\
\hline
\textbf{Erwartetes \newline Resultat:} & 
- Die neuen Einstellungen werden korrekt angewendet. \newline
- Der Bereinigungsprozess arbeitet entsprechend den benutzerdefinierten Einstellungen. \\
\hline
\end{tabular}

\vspace{1em}

\subsection*{Testfall GC04 – Duplikatentfernung}
\begin{tabular}{|p{4cm}|p{10cm}|}
\hline
\textbf{ID:} & GC04 \\
\hline
\textbf{Beschreibung:} & Test der Duplikatentfernung. \\
\hline
\textbf{Vorbedingung:} & Doppelte Dateien befinden sich im Verzeichnis. \\
\hline
\textbf{Test-Schritte:} & 
1. Erstellen von doppelten Dateien (gleicher Inhalt, unterschiedlicher Dateiname). \newline
2. Start des Bereinigungsprozesses. \newline
3. Überprüfung, ob die Duplikate entfernt werden und nur eine Datei bestehen bleibt. \\
\hline
\textbf{Erwartetes \newline Resultat:} & 
- Nur eine Kopie jeder doppelten Datei bleibt bestehen. \\
\hline
\end{tabular}

\vspace{1em}

\subsection*{Testfall GC05 – Passwortänderung}
\begin{tabular}{|p{4cm}|p{10cm}|}
\hline
\textbf{ID:} & GC05 \\
\hline
\textbf{Beschreibung:} & Überprüfung der Passwortänderungsfunktion. \\
\hline
\textbf{Vorbedingung:} & Der Benutzer ist eingeloggt, und es gibt einen Testbenutzer mit einem bekannten Passwort. \\
\hline
\textbf{Test-Schritte:} & 
1. Zugriff auf die Einstellungen, um das Passwort zu ändern. \newline
2. Eingeben des aktuellen Passworts, gefolgt von einem neuen Passwort. \newline
3. Überprüfen, ob die Passwortänderung korrekt durchgeführt wurde. \\
\hline
\textbf{Erwartetes \newline Resultat:} & 
- Das Passwort wird erfolgreich geändert. \\
\hline
\end{tabular}


\newpage
\subsection{Unit Tests}
Aktuell ist die Implementierung von Unit-Tests nicht geplant. Stattdessen wird der Fokus auf manuelle Tests gelegt, um sicherzustellen, dass die Schlüsselkomponenten des Tools wie erwartet funktionieren. Manuelle Tests werden für kritische Funktionen durchgeführt.

Eine spätere Einführung von Unit-Tests bleibt optional, abhängig von den zukünftigen Anforderungen und der Projektentwicklung.

\vspace{1em}
\textbf{Begründung:}
\begin{enumerate}
    \item \textbf{Kein Verzicht auf Qualitätssicherung:} \\
    Statt automatisierter Tests werden manuelle Tests für zentrale Funktionen priorisiert.
    
    \item \textbf{Flexibilität für die Zukunft:} \\
    Die Möglichkeit zur Einführung von Unit-Tests bleibt offen, falls sich Anforderungen oder Ressourcen ändern.
    
    \item \textbf{Zeitmanagement:} \\
    Fokus auf funktionierende Hauptfunktionen, anstatt zusätzliche Komplexität durch Tests hinzuzufügen.
\end{enumerate}
