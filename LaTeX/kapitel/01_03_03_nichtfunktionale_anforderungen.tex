Für das Garbage-Collection-Tool von DataFlow Solutions ergeben sich die folgenden nicht-funktionalen Anforderungen sowie geschätzte Entwicklungszeiten:

\begin{itemize}
  \item \textbf{Optik gemäß Corporate-Design:} Die grafische Benutzeroberfläche wird unter Einhaltung der firmeninternen Designrichtlinien entwickelt (Farbschema, Icons, Logo, Layout). \textit{(20 Stunden)}

  \item \textbf{Barrierefreie Benutzerführung:} Die WPF-Oberfläche soll auch für Nutzer mit eingeschränkter Sehfähigkeit gut nutzbar sein, z.\,B. durch kontrastreiche Darstellung, Tastaturnavigation und verständliche Statusmeldungen. \textit{(25 Stunden)}

  \item \textbf{Performance bei großen Dateibeständen:} Die Anwendung verarbeitet mehrere Tausend Dateien performant und ohne merkliche Verlangsamung der Benutzeroberfläche. Dies wird durch den Einsatz von \texttt{async} / \texttt{await} und effizienter Dateifilterung gewährleistet. \textit{(30 Stunden)}

  \item \textbf{Kompatibilität mit Windows-Systemen:} Die Software läuft zuverlässig unter Windows 10 und 11 und nutzt systemeigene Komponenten wie den Windows-Papierkorb oder den Datei-Explorer.\textit{(35 Stunden)}
\end{itemize}
