
Zusätzlich zur Standard-Bereinigung bietet das Garbage-Collection-Tool die gezielte Entfernung sogenannter „Junk“-Dateien an. Dabei handelt es sich um temporäre Dateien mit bestimmten typischen Endungen wie \texttt{.tmp}, \texttt{.log}, \texttt{.bak}, \texttt{.old}, \texttt{.dmp} oder \texttt{.swp}. Solche Dateien sammeln sich häufig in systemweiten oder anwendungsspezifischen Verzeichnissen an und können Speicherressourcen unnötig blockieren.

Die Bereinigungsfunktion wird durch den Befehl \texttt{CleanJunkFilesCommand} angestoßen, welcher in der Benutzeroberfläche über einen eigenen Button erreichbar ist. Intern ruft dieser Befehl die Methode \texttt{CleanJunkFilesAsync()} auf. Diese Methode lokalisiert Junk-Dateien im temporären Systemverzeichnis, berechnet die freigegebenen Ressourcen und löscht die Dateien, sofern sie nicht gesperrt sind.

Im folgenden Screenshot ist die zentrale Logik zur Erkennung und Löschung dieser Dateien dargestellt:

\begin{figure}[H]
    \centering
    \begin{cs}
private async Task CleanJunkFilesAsync()
{
    string tempPath = Path.GetTempPath();
    var junkFiles = Directory.GetFiles(tempPath, "*.*", System.IO.SearchOption.AllDirectories)
                                .Where(f => IsJunkFile(f))
                                .ToList();

    if (!junkFiles.Any())
    {
        StatusMessage = "Keine Junk-Dateien gefunden.";
        return;
    }

    long totalBytes = junkFiles.Sum(f => new FileInfo(f).Length);
    double spaceFreedInMb = totalBytes / (1024.0 * 1024.0);
    await LogCleanupAsync(junkFiles.Count, spaceFreedInMb, "Junk");
    await DeleteFilesAsync(junkFiles, "Löschen der Junk-Dateien");
}
\end{cs}
    \caption{Screenshot der Methode \texttt{CleanJunkFilesAsync}}
\end{figure}
% \begin{figure}[H]
%     \centering
%     \includegraphics[width=1\textwidth]{src/cleanupvm_cleanjunkfileasync_code.png}
%     \caption{Screenshot der Methode \texttt{CleanJunkFilesAsync}}
% \end{figure}

Die Erkennung basiert auf einer Hilfsmethode namens \texttt{IsJunkFile()}, die eine Liste typischer Junk-Endungen prüft und bei Übereinstimmung einen \texttt{true}-Wert zurückliefert. Diese Funktion stellt sicher, dass nur wirklich überflüssige Dateien bereinigt werden. Die folgende Abbildung zeigt die Umsetzung dieser Methode:

\begin{figure}[H]
    \centering
    \begin{cs}
private bool IsJunkFile(string filePath)
{
    string[] junkExtensions = { ".tmp", ".log", ".bak", ".old", ".dmp", ".swp" };
    return junkExtensions.Contains(Path.GetExtension(filePath).ToLower());
}
\end{cs}
    \caption{Screenshot der Methode \texttt{IsJunkFile}}
\end{figure}
% \begin{figure}[H]
%     \centering
%     \includegraphics[width=1\textwidth]{src/cleanupvm_isjunkfile_code.png}
%     \caption{Screenshot der Methode \texttt{IsJunkFile}}
% \end{figure}

Die gezielte Junk-Bereinigung stellt eine wichtige Ergänzung zur klassischen Zeit-basierten Löschung dar. Sie ermöglicht eine zusätzliche Entlastung des Systems und sorgt für eine effizientere Verwaltung temporärer Ressourcen.
