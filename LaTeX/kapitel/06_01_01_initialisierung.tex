

Die Klasse \texttt{CleanupVM} ist das zentrale \texttt{ViewModel} und das Herzstück des Programms. Sie fungiert als Vermittler zwischen der grafischen Benutzeroberfläche (View) und der Anwendungslogik (Model) und kapselt die gesamte Funktionalität rund um den Datei-Bereinigungsprozess. 

\paragraph{Eigenschaften}
Die Klasse stellt zahlreiche an die Oberfläche gebundene Eigenschaften bereit, unter anderem:

\begin{itemize}
  \item \texttt{DirectoryPath}: Der Pfad des Verzeichnisses, das bereinigt werden soll. Änderungen triggern \texttt{OnPropertyChanged}, um die UI zu aktualisieren.
  \item \texttt{OlderThanDays}: Schwellenwert in Tagen für zu löschende Dateien.
  \item \texttt{FilePatterns}: Liste mit Dateimustern (z.\,B. \texttt{*.tmp, *.log}) zur Filterung.
  \item \texttt{IntervalInMinutes}: Intervall in Minuten für den automatisierten Bereinigungsscheduler.
  \item \texttt{SchedulerStatus}, \texttt{TimeUntilNextCleanup}: Informationen zum aktuellen Scheduler-Status.
  \item \texttt{StatusMessage}, \texttt{StatusColor}: Visualisierung von Status- und Fehlermeldungen.
  \item \texttt{ProgressBarVisibility}, \texttt{ProgressValue}, \texttt{ProgressMaximum}: Steuerung der Fortschrittsanzeige.
  \item \texttt{AreButtonsEnabled}: De-/Aktivierung von Schaltflächen bei laufenden Prozessen.
\end{itemize}

\paragraph{Befehle}
Die folgenden \texttt{ICommand}-Befehle werden zur Interaktion mit der Oberfläche bereitgestellt:

\begin{itemize}
  \item \texttt{CleanupCommand}: Führt die Standard-Bereinigung aus.
  \item \texttt{CleanJunkFilesCommand}, \texttt{RemoveDuplicateFilesCommand}: Löschen von temporären Dateien bzw. Duplikaten.
  \item \texttt{StartSchedulerCommand}, \texttt{StopSchedulerCommand}: Steuerung des Schedulers.
  \item \texttt{SearchDirectoryPathCommand}: Öffnet einen Dialog zur Verzeichnisauswahl.
\end{itemize}

\paragraph{Konstruktor}
Im Konstruktor erfolgt die Initialisierung der zentralen Komponenten:

\begin{itemize}
  \item Laden der Konfiguration aus der \texttt{config.json}-Datei.
  \item Initialisierung der Befehle (\texttt{RelayCommand}).
  \item Einrichtung eines Countdowns für den nächsten automatischen Cleanup.
  \item Laden bisheriger Protokolle.
  \item Setzen von Startwerten für Statusanzeigen und Schaltflächen.
\end{itemize}


\begin{figure}[H]
    \centering
    \begin{cs}
public CleanupVM()
{
    _config = AppConfig.LoadFromJson("config.json");
    CleanupCommand = new RelayCommand(async obj => await ExecuteWithButtonDisable(CleanupAsync));
    CleanJunkFilesCommand = new RelayCommand(async obj => await ExecuteWithButtonDisable(CleanJunkFilesAsync));
    RemoveDuplicateFilesCommand = new RelayCommand(async obj => await ExecuteWithButtonDisable(RemoveDuplicateFilesAsync));
    StartSchedulerCommand = new RelayCommand(obj => StartScheduler());
    StopSchedulerCommand = new RelayCommand(obj => StopScheduler());
    SearchDirectoryPathCommand = new RelayCommand(obj => SearchDirectoryPath());
    ProgressBarVisibility = Visibility.Collapsed;

    _countdownTimer = new System.Timers.Timer(1000); 
    _countdownTimer.Elapsed += CountdownElapsed;
}
\end{cs}
    \caption{Screenshot des Konstruktors von CleanupVM}
\end{figure}
% \begin{figure}[H]
%     \centering
%     \includegraphics[width=1.0\textwidth]{src/cleanupvm_kosntruktor_code.png}
%     \caption{Screenshot des Konstruktors von \texttt{CleanupVM}}
% \end{figure}

\paragraph{Hilfsmethoden}
Neben dem Konstruktor enthält die Klasse unterstützende Methoden wie zum Beispiel:

\begin{itemize}
  \item \texttt{OnPropertyChanged}: Aktualisiert die Oberfläche bei Änderungen.
  \item \texttt{RestartTimer}: Startet den Timer neu, falls sich das Intervall ändert.
  \item \texttt{SearchDirectoryPath}: Öffnet einen Systemdialog zur Verzeichnisauswahl.
  \item \texttt{ExecuteWithButtonDisable}: Führt eine Methode aus und deaktiviert temporär die Schaltflächen.
\end{itemize}

Diese Initialisierungslogik sorgt dafür, dass die Bereinigungsfunktionen unmittelbar nach dem Laden der View voll einsatzbereit sind.