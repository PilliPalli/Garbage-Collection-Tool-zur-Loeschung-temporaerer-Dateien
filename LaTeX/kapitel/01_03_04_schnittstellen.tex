Das Garbage-Collection-Tool kommuniziert mit einer extern betriebenen MSSQL-Server-Datenbank. Diese wird aktuell in einem Linux-basierten Docker-Container ausgeführt.

Die Datenbank dient der Speicherung von Benutzerdaten, Konfigurationen sowie Protokollen über durchgeführte Bereinigungsvorgänge. Obwohl der Datenbankserver im Projektszenario lokal bereitgestellt wird, erfolgt die Anbindung netzwerkbasiert und kann daher als \textbf{externe Schnittstelle} bewertet werden.

In einem produktiven Unternehmensumfeld wäre eine ähnliche Infrastruktur üblich: Die Anwendung würde auf einem Client laufen, während die SQL-Datenbank zentral auf einem Datenbankserver innerhalb des Firmennetzwerks betrieben wird. Die Kommunikation würde ebenfalls über das Netzwerk erfolgen.
