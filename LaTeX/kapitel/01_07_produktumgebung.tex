

Für die Entwicklung und den Einsatz des Garbage-Collection-Tools wurde folgende technische Produktumgebung verwendet:

\begin{itemize} \item \textbf{Programmiersprache:}
Die Anwendung wurde in \textbf{C\#} entwickelt, da die Sprache eine umfangreiche Unterstützung für Windows-basierte Desktop-Anwendungen bietet und eine moderne objektorientierte Programmierung ermöglicht.

\item \textbf{UI-Design:}
Das Layout und die Benutzeroberfläche wurden zunächst mit \textbf{Figma} entworfen. Dieses Tool ermöglicht eine effiziente Erstellung von Prototypen und UI-Komponenten, die anschließend in die WPF-Oberfläche übertragen wurden.

\item \textbf{Entwicklungsumgebung:}
Zur Umsetzung kam \textbf{Visual Studio} zum Einsatz. Die integrierte Entwicklungsumgebung (IDE) bietet umfassende Werkzeuge für C\#-Projekte, Debugging, Versionskontrolle und Benutzeroberflächendesign.

\item \textbf{Datenbanken:}
Für die persistente Speicherung von Konfigurationsdaten, Benutzerinformationen und Bereinigungsprotokollen wird ein \textbf{SQL Server} verwendet. Dieser wird in einem Docker-Container unter Linux betrieben und ist über das Netzwerk angebunden.

\item \textbf{Betriebssysteme:}
Die Anwendung ist kompatibel mit \textbf{Windows 10} und \textbf{Windows 11}. Die Entwicklung erfolgte unter macOS mithilfe von \textbf{Parallels}, was unter einer Windows 11 VM lief.

 \end{itemize}