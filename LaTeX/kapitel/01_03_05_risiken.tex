Für das Garbage-Collection-Tool werden folgende Risiken und entsprechende Maßnahmen in Betracht gezogen:

\begin{itemize}
  \item \textbf{Technische Inkompatibilität:} Das Tool könnte auf bestimmten Systemen oder Betriebssystemversionen nicht ordnungsgemäß funktionieren.

  \textit{Maßnahme:} Durchführung umfassender Kompatibilitätstests in der Entwicklungsphase, um technische Probleme bei der Ausführung frühzeitig zu erkennen und zu beheben.

  \item \textbf{Datenverlust:} Es besteht das Risiko, dass versehentlich wichtige Dateien gelöscht werden, die weiterhin benötigt werden.

  \textit{Maßnahme:} Anstatt Dateien direkt zu löschen, werden diese zunächst in den Papierkorb verschoben. Zusätzlich sorgen Sicherheitsprüfungen (z.\,B. durch geschützte Verzeichnisse oder Dateifilter) dafür, dass keine kritischen Dateien unbeabsichtigt entfernt werden.

  \item \textbf{Geringe Nutzerakzeptanz:} Die Bedienung könnte für Endnutzer nicht intuitiv genug sein oder nicht deren Erwartungen entsprechen.

  \textit{Maßnahme:} Die Einbindung der Zielnutzer in den Entwicklungsprozess sowie regelmäßiges Einholen von Feedback erhöhen die Nutzerfreundlichkeit. Die Erstellung einer leicht verständlichen Anleitung unterstützt zusätzlich den effektiven Einsatz des Tools.
\end{itemize}
    