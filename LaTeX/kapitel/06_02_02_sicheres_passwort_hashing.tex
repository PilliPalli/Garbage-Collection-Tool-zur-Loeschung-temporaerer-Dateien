Ein besonderer Fokus lag auf der sicheren Speicherung der Passwörter. Hier fiel die Wahl auf den modernen Algorithmus Argon2. Dieser nutzt gezielt CPU- und RAM-Ressourcen, um den Aufwand für Brute-Force-Angriffe zu erhöhen. Durch frei wählbare Parameter wie Speichergröße, Iterationen und Parallelisierung kann das Sicherheitsniveau flexibel angepasst werden. Die folgende Methode implementiert das Hashing mittels des NuGet-Pakets \texttt{Konscious.Security.Cryptography}:

\begin{figure}[H]
    \centering
    \begin{cs}
private static string HashPassword(string password)
{  
    byte[] salt = new byte[16];
    RandomNumberGenerator.Fill(salt);

    var argon2 = new Argon2id(Encoding.UTF8.GetBytes(password))
    {
        Salt = salt,
        DegreeOfParallelism = 8, 
        Iterations = 4,       
        MemorySize = 65536       
    };

    byte[] hash = argon2.GetBytes(32); 

    byte[] hashBytes = new byte[48];
    Array.Copy(salt, 0, hashBytes, 0, 16);
    Array.Copy(hash, 0, hashBytes, 16, 32);

    return Convert.ToBase64String(hashBytes); 
}
\end{cs}
    \caption{Implementierung des Passwort-Hashings mit Argon2}
\end{figure}